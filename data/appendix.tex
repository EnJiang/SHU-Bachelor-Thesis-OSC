\chapter{学校模板提示(附录部分)}

论文附录依次用大写字母“附录 A、附录 B、附录 C……”表示,附录内的分级序
号可采用“附 A1、附 A1.1、附 A1.1.1”等表示,图、表、公式均依此类推为“图 A1、
表 A1、式(A1)”等。包含以下内容:

\begin{enumerate}
    \item 代码、图表、标准、手册等数据
    \item 未发表过的一手文献
    \item 公式推导与证明、调查表等
    \item 辅助性教学工具或表格
    \item 其他需要展示或说明的内容
\end{enumerate}

\chapter{经典不等式}
论文中用到的经典不等式.\\

\noindent{\bfseries (H\"older Inequality)}
设~$a_i\geq0$, $b_i\geq0$, $i=1$, $2$, $\cdots$, $n$, 且~$p>1$, $q>1$ 
满足~$1/p+1/q=1$. 则有
\[
\sum_{i=1}^{n}a_ib_i\leq\left(\sum_{i=1}^{n}a_i^p\right)^{\frac1p}
\cdot\left(\sum_{i=1}^{n}b_i^q\right)^{\frac1q},
\]
等号成立当且仅当存在一个常数~$c$ 满足~$a_i^p=cb_i^q$.\\

\noindent{\bfseries (PM Inequality)}
设~$x_1$, $x_2$, $\ldots$, $x_n$ 是~$n$ 个非负实数. 如果~$0<p<q$, 那么
\[
\left(\frac{x_1^p+x_2^p+\cdots+x_n^p}{n}\right)^{\frac{1}{p}}\leq
\left(\frac{x_1^q+x_2^q+\cdots+x_n^q}{n}\right)^{\frac{1}{q}},
\]
等号成立当且仅当~$x_1=x_2=\cdots =x_n$.\\

\noindent{\bfseries (AM-GM Inequality)}
设~$x_1$, $x_2$, $\ldots$, $x_n$ 是~$n$ 个非负实数. 则有
\[
\frac{x_1+x_2+\cdots+x_n}{n}\geq\sqrt[n]{x_1x_2\cdots x_n},
\]
等号成立当且仅当~$x_1=x_2=\cdots =x_n$.