% !Mode:: "TeX:UTF-8"
\chapter{学校模板提示(正文部分)}

正文是毕业论文的主体和核心部分,不同学科专业和不同的选题可以有不同的写作方式。正文一般包括以下几个方面。

1. 引言或背景

引言是论文正文的开端,引言应包括毕业论文选题的背景、目的和意义;对国内外研究现状和相关领域中已有的研究成果的简要评述;介绍本项研究工作研究设想、研究方法或实验设计、理论依据或实验基础;涉及范围和预期结果等。要求言简意赅,注意不要与摘要雷同或成为摘要的注解。

2. 主体

论文主体是毕业论文的主要部分,必须言之成理,论据可靠,严格遵循本学科国际通行的学术规范。在写作上要注意结构合理、层次分明、重点突出,章节标题、公式图表符号必须规范统一。论文主体的内容根据不同学科有不同的特点,一般应包括以下几个方面:

\begin{enumerate}[(1)]
    \item 毕业设计(论文)总体方案或选题的论证;
    \item 毕业设计(论文)各部分的设计实现,包括实验数据的获取、数据可行性及有效性的处理与分析、各部分的设计计算等;
    \item 对研究内容及成果的客观阐述,包括理论依据、创新见解、创造性成果及其改进与实际应用价值等;
    \item 论文主体的所有数据必须真实可靠,自然科学论文应推理正确、结论清晰;人文和社会学科的论文应把握论点正确、论证充分、论据可靠,恰当运用系统分析和比较研究的方法进行模型或方案设计,注重实证研究和案例分析,根据分析结果提出建议和改进措施等。
\end{enumerate}


3. 结论

结论是毕业论文的总结,是整篇论文的归宿。应精炼、准确、完整。着重阐述自己的创造性成果及其在本研究领域中的意义、作用,还可进一步提出需要讨论的问题和建议。
