% !Mode:: "TeX:UTF-8"
\chapter{模板介绍}
\label{cha:intro}

\section{\shuthesis 模板(本模板的基模板)介绍}
这是 \shuthesis\ 的示例文档,基本上覆盖了模板中所有格式的设置。建议大家在使用模板之
前,阅读一下 \texttt{shuthesis.pdf} 文档。\shuthesis\ 已经将 \LaTeX 的复杂性尽可
能地进行了封装,开放出简单的接口,以便于使用者可以轻易地使用。

\shuthesis\ 是为了帮助上海大学毕业生撰写学位论文而编写的 \LaTeX 模板,模板的开发
分为两个阶段:版本 v1.x 是由水寿松制作完成的,基于 CJK 宏包开发和使用 GBK 编码,
可在 \url{http://blog.lehu.shu.edu.cn/shuishousong/A209370.html} 下载。当前
版本是 v2.0, 由 ahhylau 制作完成,基于 XeCJK 宏包开发,文件使用 UTF-8 编码。
\shuthesis\ v2.0 使用文学化编程 (Literate Programming), 利用 \texttt{doc/DocStrip} 
将代码和说明文档混合编写,便于以后的升级和维护。另外,作者重新制作了上海大学 logo 的
高清矢量图,看起来更加美观。

目前 \shuthesis\ 模板的代码托管在 \href{https://github.com/ahhylau/shuthesis}{GitHub} 
上,如有修改建议或者其他要求欢迎在 GitHub 上提交 issue, 作者会尽快回复。非常期待有其
他上大的 \TeX\ 使用者加入到模板的开发与维护当中来,不断完善模板。

本模板是以清华大学学位论文模板 \textsc{ThuThesis} 为基础制作的衍生版,在此对代码的贡
献者表示感谢!

\section{\shubachelorthesisOSC\ 模板}
\shuthesis\ 仅支持硕博论文,后来 \href{https://github.com/alfredbowenfeng}{alfredbowenfeng}
在\shuthesis\ 的基础上修改出了\shubachelorthesis\ , 然而似乎格式和学习官方给出的版本有多处对不上。

因此,我们在 \href{https://github.com/alfredbowenfeng/SHU-Bachelor-Thesis}{\shubachelorthesis\ } 
的基础上进一步制作了上海大学本科生毕业论文 Latex 模板开源社区版本
\href{https://github.com/EnJiang/SHU-Bachelor-Thesis-OSC}{\shubachelorthesisOSC\ }


感谢前面几位同学的工作和开源精神。希望本模板能帮助到本科生同学,希望越来越多的同学能加入到开源社区大家庭。

\section{目录内容}
模板的源文件即为研究生毕业论文中使用的模板,用户可以通过修改这些文件来编辑自己的毕业论文。
\begin{itemize}
\item{main.tex}: 主文件,包含封面部分和基本设置。
\item{data}: 包含本文正文中的所有章节。
\begin{itemize}
\item{abstract.tex}: 中英文摘要。
\item{denotation.tex}: 主要符号对照表。
\item{chap01.tex}: 第一章内容。
\item{chap02.tex}: 第二章内容。
\item{chap03.tex}: 第三章内容。
\item{chap04.tex}: 第四章内容。
\item{acknowledgement.tex}: 致谢。
\item{publications.tex}: 作者在攻读学位期间公开发表的论文。
\item{appendix.tex}: 附录。
\end{itemize}
\item{reference/refs.bib}: 存放论文所引用的全部参考文献信息。
\item{clean.bat}: 双击此文件,可以用来清理 main.tex 在编译之后生成的所有缓存文件,
如后缀名为~.aux~,~.log~,~.bak~的文件。
\item{make-doc.bat}: 双击此文件,一键生成用户手册 \texttt{shuthesis.pdf}.
\end{itemize}


\section{模板使用}
\label{sec:first}

本模板在 Windows 10 和 \TeX Live 2016 下开发,所使用的宏包均跟进到最新版本。本模板并
未在其他平台和发行版进行测试,如 MacOS \& Mac\TeX. 由于历史原因,目前国内使用 C\TeX\ 
套装的人还是很多。然而,C\TeX\ 套装自从 2012 年后就不再更新了,许多宏包已经很老旧了。
因此从 \shuthesis\ v2.0 开始,模板不再支持在 C\TeX 套装下使用 (C\TeX\ 2.9.2 及之前
的版本均无法使用). 如果用户需要在 C\TeX\ 下写作,可使用 \shuthesis\ v1.x. 在 Windows 
系统和 Linux 系统下作者推荐使用 \TeX Live 进行编译; MacOS 系统可使用 Mac\TeX. 










